\documentclass[../main.tex]{subfiles}

\begin{document}

\subsection{FLUX: A pipeline for MEG analysis and beyond}\label{sec:FLUX}

\authors{Oscar Ferrante, %
Tara Ghafari, %
Ole Jensen}

FLUX\supercite{Ferrante2022} is an open-source pipeline for analysing magnetoencephalography (MEG) data. There are several toolboxes developed by the community to analyse MEG data. While these toolboxes provide a wealth of options for analyses, the many degrees of freedom pose a challenge for reproducible research. The aim of FLUX is to make the analyses steps and setting explicit. For instance, FLUX includes the state-of-the-art suggestions for noise cancellation as well as source modelling including pre-whitening and handling of rank-deficient data.

So far, the FLUX pipeline has been developed for MNE-Python\supercite{Gramfort2014} and FieldTrip\supercite{Oostenveld2011} with a focus on the MEGIN/Elekta system and it includes the associated documents as well as codes.
The long-term plan for this pipeline is to make it more flexible and versatile to use. One key motivation for this is to facilitate open science with the larger aim of fostering the replicability of MEG research.

These goals can be achieved in mid-term objectives, such as making the FLUX pipeline fully BIDS compatible and more automated. Another mid-term goal is to containerize the FLUX pipeline and the associated dependencies making it easier to use. Moreover, expanding the applications of this pipeline to other systems like MEG CTF, Optically Pumped Magnetometer (OPM) and EEG will be another crucial step in making FLUX a more generalized neurophysiological data analysis pipeline.

\added{During the 2022 Brainhack, the team focused on incorporating the BIDS standard into the analysis pipeline using MNE\_BIDS\supercite{Appelhoff2019mne}. Consequently, an updated version of FLUX was released after the Brainhack meeting.}

\end{document}
