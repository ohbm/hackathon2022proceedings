\documentclass[../main.tex]{subfiles}

\begin{document}

\subsection{Exploding brains in Julia}\label{sec:explodingbrains}

\authors{\"Omer Faruk G\"ulban, %
Leonardo Muller-Rodriguez}

Particle simulations are used to generate visual effects (in movies, games, etc.). In this project, we explore how we can use magnetic resonance imaging (MRI) data to generate interesting visual effects by using (2D) particle simulations. Aside from providing an entertaining avenue to the interested participants, our project has further educational utility. For instance, anatomical MRI data analysis is done in two major frameworks: (1) manipulating fixed regularly spaced points in space (also known as Eulerian point of view), and (2) manipulating moving irregularly spaced points in space (Lagrangian point of view). For instance, bias field correction is commonly done from Eulerian point of view (e.g. computing a bias field is similar to computing a particle velocity field in each frame of the explosions), whereas cortical surface inflation is commonly done from Lagrangian point of view of the MRI data (e.g. computing the inflated brain surface is similar to computing the new positions of particles in each frame of the explosion). Therefore, our project provides an educational opportunity for those who would like to peek into the deep computational and data structure manipulation aspects of MRI image analysis. We note that we already made two hackathon projects in 2020 (see below) and were first inspired by a blog post (\texttt{\url{https://nialltl.neocities.org/articles/mpm_guide.html}}) on the material point method\supercite{Jiang1965, Love2006, Stomakhin2013a}. Our additional aim in Brainhack 2022 is to convert our previous progress in Python programming language to Julia. The reason why we have moved to Julia language is because we wanted to explore this new programming language's potential for developing MRI image analysis methods as it has convenient parallelization methods that speeds-up the particle simulations (and any other advanced image manipulation algorithms).

-----------------------------------

Our previous efforts are documented at:
\begin{enumerate}
    \item 2020 OpenMR Benelux: \texttt{\url{https://github.com/OpenMRBenelux/openmrb2020-hackathon/issues/7}} 
    \item 2020 OHBM Brainhack: \texttt{\url{https://github.com/ohbm/hackathon2020/issues/124}}
    \item Available within the following github repository: \texttt{\url{https://github.com/ofgulban/slowest-particle-simulator-on-earth}}    
\end{enumerate}

-----------------------------------

As a result of this hackathon project, we delivered a video compilation of our animations (\Cref{fig:exploding_brains}) which can be seen at \texttt{\url{https://youtu.be/_5ZDctWv5X4}}. We highlight that in addition to its educational value, our project provided stress relief by means of entertaining the participants after the pandemic. We believe that our project provides a blueprint for the future brainhacks where MRI science, computation, and education can be disseminated within an engaging and entertaining context. Our future efforts will involve sophisticating the particle simulations, the initial simulation parameters to generate further variations of the visual effects, and potentially synchronizing the simulation effects with musical beats.

\begin{figure}
	\centering
	\includegraphics[width=0.5\textwidth]{images/exploding_brains.png}
	\caption{A video compilation of brain explosions can be seen at \texttt{\url{https://youtu.be/_5ZDctWv5X4}}.}
	% Add a label to reference in text. Make it specific!
	\label{fig:exploding_brains}
\end{figure}



\end{document}
