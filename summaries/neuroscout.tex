\documentclass[../main.tex]{subfiles}

\begin{document}

\subsection{Neuroscout: A platform for fast and flexible re-analysis of (naturalistic) fMRI studies}

\authors{Alejandro De La Vega, %
Roberta Rocca, %
Sam Nastase, %
Peer Horholz, %
Jeff Mentch, %
Kevin Sitek, %
Caroline Martin, %
Leonardo Muller, %
Kan Keeratimahat, %
Dylan Nielson}

Neuroscout is an end-to-end platform for analysis of naturalistic fMRI data designed to facilitate the adoption of robust and generalizable research practices. Neuroscout’s goal is to make it easy to analyze complex naturalistic fMRI datasets by providing an integrated platform for model specification and automated statistical modeling, reducing technical barriers. Importantly, Neuroscout is at its core a platform for reproducible analysis of fMRI data in general, and builds upon a set of open standards and specifications to ensure analyses are Findable, Accessible, Interoperable, and Reusable (FAIR). 

In the OHBM Hackathon, we iterated on several important projects that substantially improved the general usability of the Neuroscout platform. First, we launched a revamped and unified documentation which links together all of the subcomponents of the Neuroscout platform (https://neuroscout.github.io/neuroscout-docs/). Second, we facilitated access to Neuroscout’s data sources by simplifying the design of Python API, and providing high-level utility functions for easy programmatic data queries. Third, we updated a list of candidate naturalistic and non-naturalistic datasets amenable for indexing by the Neuroscout platform, ensuring the platform stays up to date with the latest public datasets. 

In addition, important work was done to expand the types of analyses that can be performed with naturalistic data in the Neuroscout platform. Notably, progress was made in integrating Neuroscout with Himalaya, a library for efficient voxel wide encoding modeling with support for banded penalized regression. In addition, a custom image-on-scalar analysis was prototyped on naturalistic stimuli via the publicly available naturalistic features available in the Neuroscout API. Finally, we also worked to improve documentation and validation for BIDS StatsModels, a specification for neuroimaging statistical models which underlies Neuroscout’s automated model fitting pipeline. 


\end{document}
