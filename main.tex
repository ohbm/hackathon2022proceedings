\documentclass[10pt,a4paper,twocolumns]{proc}

%--------------------------------------------
% Input and language
%--------------------------------------------
\usepackage[utf8]{inputenc}
\usepackage[T1]{fontenc}
\usepackage[spanish,german,british]{babel}

%--------------------------------------------
% License
%--------------------------------------------
\usepackage{ccicons}
\usepackage[type={CC},modifier={by-nc-sa},version={4.0},imagemodifier={-eu}]{doclicense}

%--------------------------------------------
% Graphics
%--------------------------------------------
\usepackage{graphicx}
% \usepackage[pdftex]{graphicx}
\usepackage{xcolor} % Required for specifying colors by name
\usepackage{subcaption}

%--------------------------------------------
% Hyperlinks
%--------------------------------------------
\usepackage{hyperref}
\hypersetup{
	colorlinks=true,       % false: boxed links; true: colored links
	linkcolor=black!60!red,          % color of internal links (change box color with linkbordercolor)
	citecolor=black!70!green,        % color of links to bibliography
	filecolor=black,         % color of file links
	urlcolor=black        % color of external links
}

%--------------------------------------------
% Nice tables
%--------------------------------------------
\usepackage{booktabs,multirow,makecell,dcolumn,longtable}
\newcolumntype{d}{D{+}{ \pm }{-1}}

%--------------------------------------------
% Nice equations
%--------------------------------------------
\usepackage{textgreek,amsmath,nicefrac}

%--------------------------------------------
% Images and bibliography (with oxford comma)
%--------------------------------------------
\usepackage{caption,cleveref}
\newcommand{\creflastconjunction}{, and\nobreakspace}
\usepackage[style=apa6,sorting=nyt,uniquename=false]{biblatex}
\defbibheading{bibempty}{}

\renewcommand*{\bibfont}{\footnotesize}

\newcommand\BibTeX{{\rmfamily B\kern-.05em \textsc{i\kern-.025em b}\kern-.08em
\kern-.1667em\lower.7ex\hbox{E}\kern-.125emX}}

%--------------------------------------------
% Use french spacing
%--------------------------------------------
\frenchspacing

%--------------------------------------------
% Formatting
%--------------------------------------------
\usepackage{enumitem}
\newcommand{\authors}[1]{\emph{\footnotesize #1} \\}
\newcommand{\affiliations}[1]{{\scriptsize #1} \\}


\title{Proceedings of the OHBM Brainhack 2022}
%\subtitle{subtitle}

% Abilitate subfiles
\usepackage{subfiles}

% Bibliography & figures
\graphicspath{{images/}}
\addbibresource{\subfix{references/ahead.bib}}
\addbibresource{\subfix{references/brainhackcloud.bib}}
\addbibresource{\subfix{references/datalad-dataverse.bib}}
\addbibresource{\subfix{references/exploding_brains.bib}}
\addbibresource{\subfix{references/flux.bib}}
\addbibresource{\subfix{references/hyppomriqc.bib}}
\addbibresource{\subfix{references/metadata-community.bib}}
\addbibresource{\subfix{references/narps-open-pipelines.bib}}
\addbibresource{\subfix{references/neurocausal.bib}}
\addbibresource{\subfix{references/physiopy-documentation.bib}}
\addbibresource{\subfix{references/rba.bib}}
\addbibresource{\subfix{references/VASOMOSAIC.bib}}

\begin{document}

\maketitle

\authors{Stefano Moia\textsuperscript{1, 2, 3}, %
Hao-Ting Wang\textsuperscript{1, 4}, %
Anibal S. Heinsfeld\textsuperscript{5, 6}, %
Dorota Jarecka\textsuperscript{7}, %
Yu-Fang Yang\textsuperscript{8}, %
Stephan Heunis\textsuperscript{9}, %
Michele Svanera\textsuperscript{10}, %
Benjamin De Leneer\textsuperscript{1, 11, 12, 13}, %
Andrea Gondova\textsuperscript{1, 14, 15}, %
Sin Kim\textsuperscript{16}, %
Arshitha Basavaraj\textsuperscript{17}, %
Johanna M. Bayer\textsuperscript{1, 18, 19}, %
Roza G. Bayrak\textsuperscript{1, 20}, %
Pierre-Louis Bazin\textsuperscript{21, 22}, %
Isil Poyraz Bilgin\textsuperscript{4, 23}, %
Steffen Bollmann\textsuperscript{24, 25}, %
Daniel Borek\textsuperscript{26}, %
Valentina Borghesani\textsuperscript{1, 27}, %
Trang Cao\textsuperscript{28}, %
Gang Chen\textsuperscript{29}, %
Alejandro De La Vega\textsuperscript{30}, %
Sebastian Dresbach\textsuperscript{31}, %
Philipp Ehses\textsuperscript{32}, %
Jan Ernsting\textsuperscript{33, 34, 35}, %
Inês Esteves\textsuperscript{36}, %
Oscar Ferrante\textsuperscript{37}, %
Kelly G. Garner\textsuperscript{38}, %
Rémi Gau\textsuperscript{39}, %
Elodie Germani\textsuperscript{40}, %
Tara Ghafari\textsuperscript{37}, %
Satrajit S. Ghosh\textsuperscript{7, 41}, %
Sarah Elizabeth Goodale\textsuperscript{42}, %
Cassandra D. Gould van Praag\textsuperscript{43, 44}, %
Samuel Guay\textsuperscript{45}, %
Omer Faruk Gulban\textsuperscript{46, 31}, %
Yaroslav O. Halchenko\textsuperscript{47}, %
Michael Hanke\textsuperscript{9, 48}, %
Peer Herholz\textsuperscript{49}, %
Katja Heuer\textsuperscript{50}, %
Felix Hoffstaedter\textsuperscript{9, 51}, %
Ruoqi Huang\textsuperscript{52}, %
Renzo Huber\textsuperscript{53}, %
Ole Jensen\textsuperscript{37}, %
Kan Keeratimahat\textsuperscript{54}, %
Julian Q. Kosciessa\textsuperscript{55}, %
Sladjana Lukic\textsuperscript{56}, %
Neville Magielse\textsuperscript{9, 57, 58}, %
Christopher J. Markiewicz\textsuperscript{59}, %
Caroline G. Martin\textsuperscript{60}, %
Camille Maumet\textsuperscript{61}, %
Anna Menacher\textsuperscript{62}, %
Jeff Mentch\textsuperscript{7, 63}, %
Christian Mönch\textsuperscript{9}, %
Shammi More\textsuperscript{9, 48}, %
Leonardo Muller-Rodriguez\textsuperscript{64, 65}, %
Samuel A. Nastase\textsuperscript{66}, %
Eliana Nicolaisen-Sobesky\textsuperscript{9}, %
Dylan M. Nielson\textsuperscript{67}, %
Christopher R. Nolan\textsuperscript{38}, %
François Paugam\textsuperscript{68, 69}, %
Pedro Pinheiro-Chagas\textsuperscript{70}, %
Ana Luísa Pinho\textsuperscript{71, 72}, %
Alessandra Pizzuti\textsuperscript{53, 46}, %
Benjamin Poldrack\textsuperscript{64}, %
Benedikt A. Poser\textsuperscript{53}, %
Roberta Rocca\textsuperscript{73}, %
Jacob Sanz-Robinson\textsuperscript{74}, %
Kelvin Sarink\textsuperscript{34}, %
Kevin R. Sitek\textsuperscript{75}, %
Nadine Spychala\textsuperscript{76}, %
Rüdiger Stirnberg\textsuperscript{32}, %
Michał Szczepanik\textsuperscript{9}, %
Mohammad Torabi\textsuperscript{77}, %
Roberto Roberto Toro\textsuperscript{50}, %
Sebastian GW Urchs\textsuperscript{78}, %
Sofie L. Valk\textsuperscript{22, 9, 81}, %
Adina S. Wagner\textsuperscript{9, 80}, %
Laura K. Waite\textsuperscript{9}, %
Alexander Q. Waite\textsuperscript{9}, %
Lea Waller\textsuperscript{81}, %
Tyler James Wishard\textsuperscript{82, 83}, %
Jianxiao Wu\textsuperscript{9, 48}, %
Yuchen Zhou\textsuperscript{84}, %
Janine D. Bijsterbosch\textsuperscript{1, 85}, %
The physiopy community\textsuperscript{86}, %
}
\\
\affiliations{1. OHBM Open Science Special Interest Group (SIG); %
2. Neuro-X Institute, École polytechnique fédérale de Lausanne, Geneva, Switzerland; %
3. Department of Radiology and Medical Informatics (DRIM), Faculty of Medicine, University of Geneva, Geneva, Switzerland; %
3. Centre de recherche de l'Institut universitaire de gériatrie de Montréal, Montréal, QC, Canada; %
4. Centre de recherche de l'Institut universitaire de gériatrie de Montréal, Montreal, Quebec, Canada; %
5. Department of Psychology, Center for Perceptual Systems, The University of Texas at Austin, Austin, TX, USA; %
6. Department of Computer Science, The University of Texas at Austin, Austin, TX, USA; %
7. McGovern Institute for Brain Research, Massachusetts Institute of Technology, Cambridge, MA, USA; %
8. Division of Experimental Psychology and Neuropsychology, Department of Education and Psychology, Freie Universität Berlin, Berlin, Germany; %
9. Institute of Neuroscience and Medicine, Research Centre Jülich, Jülich, Germany; %
10. Centre for Cognitive Neuroimaging, School of Psychology & Neuroscience University of Glasgow, UK; %
11. Institute of Biomedical Engineering, Polytechnique Montreal, Montreal, QC, CA; %
12. CHU Sainte-Justine Research Center, University of Montreal, Montreal, QC, CA; %
13. Department of Computer Engineering and Software Engineering, Polytechnique Montreal, Montreal, CA; %
14. UNIACT, NeuroSpin, CEA, Université Paris-Saclay, Paris, France; %
15. NeuroDiderot, Inserm, Université Paris Cité, Paris, France; %
16. Brain and Cognitive Engineering, Korea Advanced Institute of Science and Technology, Daejeon, Korea; %
17. Data Science and Sharing Team (DSST), Intramural Research Program, NIMH, Bethesda, USA; %
18. Centre for Youth Mental Health, the University of Melbourne, Australia; %
19. Orygen, Melbourne, Australia; %
20. Department of Computer Science, Vanderbilt University, Nashville TN, USA; %
21. University of Amsterdam, Amsterdam, Netherlands; %
22. Max Planck Institute for Human Cognitive and Brain Science, Leipzig, Germany; %
23. Machine Learning Department and the Neuroscience Institute, Carnegie Mellon University, Pittsburgh, US; %
24. School of Information Technology and Electrical Engineering, The University of Queensland, Brisbane, Australia; %
25. Centre for Innovation in Biomedical Imaging Technology, University of Queensland, Brisbane, Australia; %
26. Department of Data Analysis, Faculty of Psychology and Educational Sciences, Ghent University, Ghent, Belgium; %
27. Department of Psychology, Faculty of Psychology and Science of Education, University of Geneva, Geneva, Switzerland; %
28. School of Psychological Sciences, Monash University, Clayton, VIC, Australia; %
29. Scientific and Statistical Computing Core, NIMH, NIH, Department of Health and Human Services, USA; %
30. Department of Psychology, University of Texas at Austin, Austin, TX, USA; %
31. Maastricht University, Maastricht, Netherlands; %
32. German Center for Neurodegenerative Diseases (DZNE), Bonn, Germany; %
33. Institute for Geoinformatics, University of Münster, Germany; %
34. University of Münster, Institute for Translational Psychiatry, Münster, Germany; %
35. Faculty of Mathematics and Computer Science, University of Münster, Germany; %
36. ISR-Lisboa and Department of Bioengineering, Instituto Superior Técnico – Universidade de Lisboa, Lisbon, Portugal; %
37. Centre for Human Brain Health, School of Psychology, University of Birmingham, UK; %
38. School of Psychology, The University of New South Wales, NSW, Australia; %
39. Institut de recherche en sciences psychologique, Université catholique de Louvain. Louvain la neuve, Belgique; %
40. Univ Rennes, Inria, CNRS, Inserm, Rennes, France; %
41. Department of Otolaryngology - Head and Neck Surgery, Harvard Medical School, Boston, MA, USA; %
42. Department of Biomedical Engineering, Vanderbilt University; %
43. Alan Turing Institute, London, UK; %
44. Department of Psychiatry, Oxford University, Oxford, UK; %
45. Department of Psychology, University of Montreal, Canada; %
46. Brain Innovation, Maastricht, The Netherlands; %
47. Department of Psychological and Brain Sciences, Dartmouth college, New Hampshire, USA; %
48. Medical Faculty, Heinrich Heine University Düsseldorf; %
49. McGill, Montreal Neurological Institute - Hospital, Montreal, Quebec, Canada; %
50. Institut Pasteur, Université Paris Cité, Unité de Neuroanatomie Appliquée et Théorique, F-75015 Paris, France; %
51. Institute of Systems Neuroscience, Medical Faculty & University Hospital Düsseldorf, Heinrich-Heine- University, Düsseldorf, Germany; %
52. University of Toronto, Toronto, Ontario, Canada; %
53. Department of Cognitive Neuroscience, Faculty of Psychology and Neuroscience, Maastricht University, Maastricht, The Netherlands; %
54. Nuffield Department of Population Health, University of Oxford, Oxford, UK; %
55. Donders Institute for Brain, Cognition and Behaviour, Radboud University, Nijmegen, The Netherlands; %
56. Ruth S. Ammon College of Education and Health Sciences, Department of Communication Sciences and Disorders, Adelphi University, San Francisco, CA, USA; %
57. Otto Hahn Cognitive Neurogenetics Group, Max Planck Institute for Human Cognitive and Brain Sciences, Stephanstraße 1A, 04103, Leipzig, Germany; %
58. Institute of Systems Neuroscience, Medical Faculty, Heinrich Heine University Düsseldorf, Moorenstraße 5 40225 Düsseldorf, Germany; %
59. Stanford University, CA, USA; %
60. Vanderbilt University, Nashville, TN USA; %
61. Inria, Univ Rennes, CNRS, Inserm, Rennes, France; %
62. Department of Statistics, University of Oxford, UK; %
63. Program in Speech and Hearing Bioscience and Technology, Harvard University, Cambridge, MA, USA; %
64. Psychoinformatics Lab, Forschungszentrum Julich, Julich, Germany; %
65. Cambridge Institute for Music Therapy Research (CIMTR), Anglia Ruskin University, Cambridge, UK; %
66. Princeton Neuroscience Institute, Princeton University, Princeton, NJ, USA; %
67. Machine Learning Team, National Institute of Mental Health, Bethesda, MD, USA; %
68. University of Montreal, Montreal, Quebec, Canada; %
69. Mila - Quebec Artificial Intelligence Institute, University of Montreal, Montreal, Quebec, Canada; %
70. Department of Neurology, Memory and Aging Center, University of California, San Francisco, San Francisco, CA, USA; %
71. Western Institute for Neuroscience, Western University, London, Ontario, Canada; %
72. Department of Computer Science, Faculty of Science, Western University, London, Ontario, Canada; %
73. Department of Culture, Cognition, and Computation, School of Culture and Society, Aarhus University, Aarhus, Denmark; %
74. Neuroscience, McGill University, Quebec, Canada; %
75. Department of Communication Science and Disorders, School of Health and Rehabilitation Sciences, University of Pittsburgh, Pittsburgh, PA, USA; %
76. Department of Informatics, University of Sussex, UK; %
77. Graduate Program in Biological and Biomedical Engineering, McGill University, Montreal, Canada; %
78. NeuroDataSciene - ORIGAMI lab, Montreal Neurological Institute-Hospital, McGill University, Montreal, Canada; %
79. Institute of Systems Neuroscience Heinrich Heine University Düsseldorf; %
80. Department of Psychology, Heinrich Heine University Düsseldorf, Düsseldorf, Germany ; %
81. Charité Universitätsmedizin Berlin, corporate member of Freie Universität Berlin and Humboldt-Universität zu Berlin, Department of Psychiatry and Neurosciences CCM, Berlin, Germany; %
82. Department of Psychiatry and Biobehavioral Sciences, University of California Los Angeles, Los Angeles, California, United States; %
83. Department of Neurology, Washington University in St. Louis, St. Louis, Missouri, United States; %
84. Department of Psychology, Carnegie Mellon University, Pittsburgh, USA; %
85. Department of Radiology, Washington University in St Louis, Saint Louis, Missouri, USA; %
physiopy.github.io %
}
\\

\begin{abstract}
OHBM Brainhack 2022 took place in June 2022. The first hybrid OHBM hackathon, it had an in-person component taking place in Glasgow and three hubs around the globe to improve inclusivity and fit as many timezones as possible.
In the buzzing setting of the Queen Margaret Union and of the virtual platform, 23 projects were presented for development.
Following are the reports of 14 of those, as well as a recapitulation of the organisation of the event. 
\end{abstract}

\section*{Introduction}

The Organisation of Human Brain Mapping BrainHack (shortened to OHBM
Brainhack in the article) is a yearly satellite event of the main OHBM
meeting, organised by the Open Science Special Interest Group following
the model of Brainhack hackathons \parencite{Gau2021}.
Where other hackathons set up a competitive environment based on
outperforming other participants' projects, Brainhacks fosters a
collaborative environment in which participants can freely collaborate
and exchange ideas within and between projects.

This edition of the OHBM Brainhack, that ran across the world over four
days, was particularly special for two reasons: it celebrated the tenth
year anniversary of Brainhack, and, like the main OHBM conference, it
was the first edition to feature an in-person event after two years of
virtual events. For this reasons, the whole organisation rotated around
five main principles:

\begin{enumerate}
\tightlist
\item
  Providing a hybrid event incorporating the positive aspects of
  in-person and virtual events alike,
\item
  Celebrating the 10\textsuperscript{th} anniversary of the
  Brainhack by bringing back newcomer-friendly hands-on hacking and
  learning experience, enhancing the Hacktrack and formatting the
  Traintrack as a collection of materials to consult beforehand and as
  spontaneous meetings of the participants aimed to learn together,
\item
  Bridging the gap between the Brainhack community and the main
  neuroimaging software developer groups, e.g. AFNI, FSL, SPM,
\item
  Due to amount of work required to meet the previous three principles,
  incorporating from the beginning a team of core organisers with a
  democratic approach to organisation, with a member in charge of an
  aspect of the event,
\item
  Brainhack event organisation should always be experimental, trying
  different solutions and formats to find a way to improve Brainhack
  events overall. 
\end{enumerate}

After a quick explanation of each main contribution of the core team, the next pages are dedicated to the summaries of the projects that were developed during the four days of hacking.

\section{Hacktrack}

\authors{Dorota Jarecka, %
Yu-Fang Yang, %
Hao-Ting Wang, %
Stefano Moia}
%

The key component of each Brainhack is hacking. The hacking part, known as
hacktrack, is where attendees collaborate on projects and explore their
own ideas. There are 4 elements of hacktrack that were organised:
project submission, project pitch, hacking period and project summary.
For the project submission, we used the GitHub issue submission process
that was used during recent years. We updated and simplified a
project template from previous years and asked project leaders to open
an issue for each project. Each issue after quick check was approved by
the moderators and automatic workflows written by the team were
responsible for sending project descriptions to the Brainhack page and
setting Discord's channels. We received 38 projects that were submitted
using this system. The project pitch was set for the morning of the
first day and everyone had 2 minutes to talk about the suggested project
and possible collaborations. After the pitches people had a chance to
talk to each other and join the projects they were interested in. This
year, we tried to maximise the time for hacking by providing a sparse
schedule for talks. The closing ceremony of the Brainhack featured 23
project reports, in which teams talked about their experiences and
described the work they accomplished.

This edition we allowed remote attendance from other locations. We
organised three hubs aiming to cover all time zones, including 1)
Asia-Pacific, 2) Glasgow, Europe, Middle East, and Africa, and 3) the
Americas, to foster inclusiveness in the hybrid conference format. We also
ensured that each hub had one live streamed session with the physical
hub in Glasgow.

\section{Traintrack}
\authors{Yu-Fang Yang, %
Dorota Jarecka}
%

Traintrack is the educational component of Brainhack events. The aim is
to introduce tools and skills for attendees to start hacking. Unlike
conventional scientific educational workshops centred around lectures
and talks, data science skills are better learned through hands-on
experience than lectures. With the Brainhack community growing mature,
the community has developed their own curated educational material.
\href{https://school.brainhackmtl.org/}{\emph{Brainhack School}} has
supplied high-quality content for independent study on a variety of
themes.

This year, we combine the collaborative nature of brainhack projects and
educational content to reimagine the format of traintack. Thus, we
replaced tutorial lectures in the previous editions with curated online
educational contents, released them prior to the main event, and
attempted to integrate them with the hacktrack projects. This format
also provides more time (i.e. schedule) and space (i.e. minimising large
space not used for hacking) for attendees to self-organise. Participants
were encouraged to form study groups on five suggested topics: 1)
setting up your system for analysis 2) python for data analysis, 3)
machine learning for neuroimaging, 4) version control systems, 5) cloud
resource. The curated content was advertised on the main hackathon
website. One dedicated channel was created on the hackathon Discord
server. Individuals could determine the nature of their experiences and
the skills they liked to acquire. Participants could form their own
study group and on any selected topic. We would like to continue the
experimentation on this format in the coming year.

\section{Platforms, website, and IT}
\authors{Anibal Solon Heinsfeld}
%

Trying to bring a positive experience for both virtual and in-person
attendees, we implemented several integrated solutions to ease
communication in the different phases of the Hackathon, focusing on a
single platform for the main event.

The first solution was the project's advertisement, in which the
community promotes their projects, the goals for the Hackathon, and
relevant information to get people interested and set to collaborate. To
do so, we used the Github Issues feature in the Hackathon
repository as the entrance for projects. Github Issues has been proven
to be accepted by the community that relies on Github for code
versioning, and was a successful approach in past hackathons.

In this edition, we were able to use Github Issue forms, a beta feature
in Github. Past use of issues for project registration relies on
Markdown code to specify which information the hacker needs to provide.
However, the code can be easily broken and changed, which makes it
harder to parse the information in automated setups. Towards this issue,
the Issue Form can lower the barrier when submitting a project. By
specifying form fields for the participants to fill, they faced a common
web form instead of a Markdown editor, bringing more structure to their
inputs and not requiring them to write code. After the organisers' quick
validation, the project information was provided to the rest of the
system. Per an automated pipeline, this information was compiled into
the website.

The second solution was the central platform for real-time
communication, namely Discord. For the first time using the platform for
an OHBM Hackathon, Discord showed potential in bringing an all-in-one
solution. Its track record with different communities and their formats
was an essential prospect for the success of a hybrid hackathon,
together with the different ways of communicating provided by the
platform. Specifically, Discord offered chat and audio/video channels,
with fine-tuned controls on permissions to see a channel, speak and use
the camera, and send messages. With these features, we were able to
create experiences for the attendants, such as text channels for
consolidating information about the hackathon, main stages controlled by
the hub hosts, a channel to join projects and hubs, and integrated text
\& voice channels for each project. The main stage was connected to a
laptop in the venue, providing synchronous streaming for announcements,
project pitches and progress reports for those participating virtually.
The project channels were automatically created together with the Github
Issues. However, given the thriving number of projects, the Discord
server was replete with project channels. Such a scenario was
overwhelming for the attendants, especially for those approaching
Discord for the first time. To ameliorate this issue, a main projects
channel was created, so attendants could automatically join projects via
related emoji reactions. The project channels were of public access;
however, only displayed upon joining the project. Besides initial
technical hiccups, the platform proved a good alternative for such an
event format.

These integrated solutions smoothed the organisation of the event, the
virtual platform provided great support for the on-line participants.
However, there was not a lot of interaction between in-person and online
participants, and projects were mainly either virtual or in-person (with
few exceptions). This is probably because hybrid hacking provides
challenges for organisation and attendants alike, even just in the
physical limitations of being able to have a video conference with a
split team. It is important to consider, however, that this was also the
first in-person event for many participants, who preferred in-person
interaction and collaboration rather than the same on-line interaction
that characterised such events in the previous two years.

\section{Project Reports}

In total, 23 projects were presented at the Brainhack, of which 14 submitted a written
report.

\subfile{summaries/ahead-project.tex}
\subfile{summaries/brainhack-cloud.tex}
\subfile{summaries/datalad-catalog.tex}
\subfile{summaries/datalad-dataverse.tex}
\subfile{summaries/exploding_brains.tex}
\subfile{summaries/flux.tex}
\subfile{summaries/hyppomriqc.tex}
\subfile{summaries/metadata-community.tex}
\subfile{summaries/narps-open-pipelines.tex}
\subfile{summaries/neurocausal.tex}
\subfile{summaries/neuroscout.tex}
\subfile{summaries/physiopy-documentation.tex}
\subfile{summaries/rba.tex}
\subfile{summaries/VASOMOSAIC.tex}

\section{Conclusion and future directions}
\authors{Stefano Moia, %
Hao-Ting Wang}
%

Approaching the organisation of an event as an experiment allows
incredible freedom and dynamicity, albeit knowing that there will be
risks and venues of improvement for the future.

The organisation managed to provide a positive onsite environment,
aiming to allow participants to self-organise in the spirit of the
Brainhack \parencite{Gau2021}, with plenty of moral - and physical - support.

The technical setup, based on heavy automatisation flow to allow project
submission to be streamlined, was a fundamental help to the organisation
team, that would benefit even more from the improvement of such
automatisation flows.

This year, representatives of AFNI, FSL, and SPM (among the major
neuroscience software developers) took part in the event, and their
presence was appreciated both by other participants and themselves. In
the future, connecting to more developers, not only from the MRI
community, might improve the quality of the Brainhack even more.

The most challenging element of the organisation was setting up an
hybrid event. While this element did not go as smoothly as it could
have, this experimental setup seemed to have worked, allowing the
participation of about 70 participants online. However, there is still a
lot to improve for a truly hybrid event. For instance, it is important
to allow spaces (both in time and space) for participants on-site to
interact with online participants, and more attention, time, volunteers,
and equipment has to be put to achieve a smooth online participation.
For this reason, the Open Science Special Interest Group instituted a
position to have a dedicated person for the hybridisation process. The
other challenge was to welcome newcomers into this heavily
project-development-oriented event. While newcomers managed to
collaborate with projects and self-organise to learn open science
related skills, this integration of pre-event train track and beginner
friendly process will benefit from more attention.

Overall this HBM Brainhack was a successful outcome for the organisation
team experiment, and we hope that our findings will be helpful to future
Brainhack events organisations.

\printbibliography

\end{document}
